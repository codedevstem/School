\documentclass{article}
\usepackage[english]{babel}
\usepackage{url}
\usepackage{graphicx}
\usepackage{amsthm}
\usepackage{hyperref}
\begin{document}
	\begin{titlepage} 
		\centering
		{\scshape\Large Dikult 105\par}
		\vspace{2em}
		{\scshape\large Candidate Number: 137\par}
		\vspace{6em}
		{\scshape\Large Assignment number?\par}
		{\scshape\LARGE A clash of colours.\par}
		\vspace{1em}
		{\scshape\url{http://designconference.aiga.org/#!/}\par}
		\vfill
		{\scshape Word count: 0000\par}
		\vspace{3em}
		\large\today
	\end{titlepage}
	{\centering
        \includegraphics[width=\textwidth,height=\textheight,keepaspectratio]{frontpage}
    }
    \vspace{5em}
	\section{Preface}
        Balanced, and yet not. Controlled, but flawed. Sleek, but stressful. These are some of my first impressions when entering this website. In the words of Steve Jobs - "Design is not just what it looks like and feels like. Design is how it works."\\ And I agree, but it is important how it looks and feels; and I am sceptical at what i am seeing and I will explain more on this in the design-section. Working my way through this sites navigation I will discover interesting solutions and consequences on making design for mobile that crashes with usage on other platforms.
    \section{Introduction}
        The Aiga conference website is part of the awwwards collection of websites using unusual navigation. This analysis will discuss some aspects of the  \url{https://www.awwwards.com/websites/unusual-navigation/} and I wanted to express my 
    \section{Analysis}
        \subsection{title}
    \section{Conclusion}
    
\end{document}