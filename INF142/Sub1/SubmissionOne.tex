\documentclass{article}
\usepackage[english]{babel}
\usepackage{url}
\begin{document}
	\begin{titlepage}
		\centering
		{\scshape\Large INF142 submission one.\par}
		\vspace{3em}
		{{\scshape\large By Kristian Os \par}}
		\vfill
		\large\today
	\end{titlepage}
	\tableofcontents
	\pagebreak
	
	\section{DNSSEC}
		\subsection{Definitions}
		DNS is an abbreviation for Domain Name System,
		and DNSSEC for Domain Name System Security Extensions. 
		
		\subsection{What does DNS do?}
		To reach another node on the Internet there must be denoted a number/name as an address. It must be unique so that one address only can point to one node.\\
		ICANN manages these so that each node has one of these unique addresses. For the same reason as we do not reference our devices based on their MAC-addresses, DNS translates the name version of the address to a number. This makes it much easier to remember the different number versions of the address.
		
		\subsection{Security}
		Within the DNS there are no security measures that can stand to the modern attacks that are getting more complex and malicious by the day. Recently a vulnerabilities in the DNS were found that allowed an attacker to 
		
	
	\clearpage
	\section{Two-factor authentication}
		\subsection{Definition}
	\nocite{*}
	\bibliography{mainBib}
	\bibliographystyle{plain}

\end{document}