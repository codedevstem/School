\documentclass{article}
\usepackage[english]{babel}
\usepackage{url}
\usepackage{graphicx}
\usepackage{amsthm}
\usepackage{hyperref}
\usepackage[round, authoryear]{natbib}
\bibliographystyle{chicago}

\begin{document}
	\begin{titlepage} 
		\centering
		{\scshape\Large Dikult 105\par}
		\vspace{2em}
		{\scshape\large Candidate Number: 137\par}
		\vspace{6em}
		{\scshape\Large Assignment number?\par}
		{\scshape\LARGE A clash of colours.\par}
		\vspace{1em}
		{\scshape\url{http://designconference.aiga.org/#!/}\par}
		\vfill
		{\scshape Word count: 0000\par}
		\vspace{3em}
		\large\today
	\end{titlepage}
	{\centering
        \includegraphics[width=\textwidth,height=\textheight,keepaspectratio]{frontpage}
    }
    \vspace{5em}
	\section{Preface}
	    Balanced, and yet not. Controlled, but flawed. Sleek, and stressful. These are some of my first impressions when entering this website. Steve Jobs said in an interview one while talking about design that "It's not just what it looks like and feels like. Design is how it works."\citep*{SteveJobsQuote} And I agree, but it is important how it looks and feels; and I am sceptical at what i am seeing. This will be explain more in the design-section of the analysis. Working my way through this sites navigational elements I will show how they try to give information in an elegant and quick way, but always struggling to make me enjoy it.
    \section{Introduction}
	    The AIGA (The American Institute of Graphic Arts) conference website is part of the awwwards collection of websites using unusual navigation. Aiga is a community of design advocates and practitioners where members can learn, help, share and create content that makes the design community thrive. The website I have chosen is for the yearly conference that AIGA hosts, where members can go to and listen to selected speakers and gather as a community. The AIGA conference that was hosted this year in October was a milestone for the Institute. Therefore this website was clearly made for drawing members of AIGA to the conference and give them the information they need/want to have. I chose this site then because the way it cached my eye and the chance of analysing a website made by designers, for designers. 
    \section{The analysis}
	    \subsection{The squid in the room}
		    There are many things to discuss and analyse about this website and the first object on the agenda is the huge abstract figure flowing down the screen. "A focal point is any element on a page that draws the viewer’s eye" \citep[Page. 22]{PWebDesign}. And what might be a just some web-designers spontaneous idea for a front page and really threw me off balance, is surely one of the design features, working as a focal point, that actually made me stay on the website rather than discard it like all the others. More than giving the front page more than something to look at, the figure in the background might represent the floating and evolving of web design as a whole. This complements the way the text is structure. Just like the figure the text comes in gradually and announces itself. Giving it importance and since this is a conference it fits nicely to the theme. Giving it the structure the text has gives meaning with the different proximity of the letters. Here the different words gives their own focal point and gives the feeling of reading words rather than just scattered letters.      
        \subsection{The Layout}
	     While the front page of this website is very engaging and exiting. I feel that it gets a bit too much the fifth time i visit the site.
	     
	     symbol of the ever changing world of web design.
    \section{Conclusion}
	\bibliography{references}
    
    
\end{document}